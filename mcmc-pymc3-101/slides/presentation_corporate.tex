% Title:
% 	Corporate presentation template
% ----------------------
% Description:
% 	A presentation template which does not look too academic.
%
% Creator: Tommy O.

% -------------------------------------------------------------------------
% Setup
% -------------------------------------------------------------------------
\documentclass[12pt, aspectratio=149]{beamer}
% Options for aspectratio: 1610, 149, 54, 43 and 32, 169
\usepackage[utf8]{inputenc}
\usepackage[norsk]{babel}% Alternative: 'norsk'
\usepackage[expansion=false]{microtype}% Fixes to make typography better
\usecolortheme{beaver} % Decent options: beaver, rose, crane
%\useoutertheme{split}
%\useoutertheme[footline=authortitle]{miniframes}
\usepackage{txfonts}% Font which looks less 'academic'
\usepackage{listings}% To include source-code
\usepackage{booktabs}% Professional tables
\usefonttheme{serif}
\usepackage{mathptmx}
\usepackage[scaled=0.9]{helvet}
\usepackage{courier}

\title{An introduction to probabilistic programming with PyMC3}
%\subtitle{Presentation subtitle}
%\institute{Inmeta}
\date{\today}
\author{Sean Meling Murray and Solveig Masvie}

% -------------------------------------------------------------------------
% Package imports
% -------------------------------------------------------------------------
\usepackage{etoolbox}
\usepackage{graphicx}
\usepackage{tikz}
\usepackage{amsmath}
\usepackage{amsthm}
\usepackage{amsfonts}
\usepackage{amssymb}
\usepackage{mathtools}
\usepackage{graphicx}
\usepackage{hyperref}
\usepackage{listings}
\usepackage[sharp]{easylist}
\usepackage{multicol}
\usepackage{tikz-cd}
\usepackage{xcolor}


%gets rid of bottom navigation bars
\setbeamertemplate{footline}[frame number]{}

%gets rid of bottom navigation symbols
\setbeamertemplate{navigation symbols}{}

% Set up colors to be used
\definecolor{purered}{RGB}{204,0,0}
\definecolor{titlered}{RGB}{229,78,71}
\definecolor{bggray}{RGB}{242,242,242}
\definecolor{bggraydark}{RGB}{217,217,217}

% Change the default colors

\setbeamercolor*{title}{bg=bggray,fg=titlered}
\AtBeginEnvironment{theorem}{%
	\setbeamercolor{block title}{fg=titlered, bg=bggraydark}
	\setbeamercolor{block body}{fg=black,bg=bggray}
}
\AtBeginEnvironment{proof}{%
	\setbeamercolor{block title}{bg=bggraydark}
	\setbeamercolor{block body}{fg=black,bg=bggray}
}
\AtBeginEnvironment{example}{%
	\setbeamercolor{block title example}{bg=bggraydark}
	\setbeamercolor{block body example}{fg=black,bg=bggray}
}
\AtBeginEnvironment{definition}{%
	\setbeamercolor{block title}{bg=bggraydark}
	\setbeamercolor{block body}{fg=black,bg=bggray}
}

\setbeamercolor{block title example}{bg=bggraydark}
\setbeamercolor{block body example}{fg=black,bg=bggray}
\setbeamercolor{block title}{bg=bggraydark}
\setbeamercolor{block body}{fg=black,bg=bggray}

\setbeamercolor{frametitle}{fg=titlered,bg=bggray}
\setbeamercolor{section in head/foot}{bg=black}
\setbeamercolor{author in head/foot}{bg=black}
\setbeamercolor{date in head/foot}{fg=titlered}

% Custom mathematics commands
\DeclareMathOperator{\R}{\mathbb{R}}
\DeclareMathOperator{\Q}{\mathbb{Q}}
\DeclareMathOperator{\Z}{\mathbb{Z}}
\DeclareMathOperator{\N}{\mathbb{N}}
% \DeclareMathOperator{\C}{\mathbb{C}}

% Spacing for lsits
\newcommand{\listSpace}{0.2em}

% Theorems, equations, definitions setup
\theoremstyle{plain}

% -------------------------------------------------------------------------
% Document start
% -------------------------------------------------------------------------
\begin{document}
\maketitle
\addtobeamertemplate{frametitle}{}{%
\begin{tikzpicture}[remember picture,overlay]
  	\node[anchor=south west,yshift=0pt, xshift=4pt] at (current page.south west) {\includegraphics[height=0.5cm]{figs/inmeta.png}};
\end{tikzpicture}}
  
\begin{frame}{Table of contents}
	\tableofcontents
\end{frame}

% -------------------------------------------------------------------------
\section{Introduction}
\begin{frame}[fragile]{Road map}
	\begin{easylist}[itemize]
		\ListProperties(Space=\listSpace, Space*=\listSpace, Style2*=$\color{bggraydark}\blacktriangleright$\space)
		# Theory
		## The basics of Bayesianism
		## Markov chain Monte Carlo methods (MCMC)
		# Practice
		## Probabilistic programming with PyMC3
	\end{easylist}
\end{frame}

\begin{frame}[fragile]{What is Bayesian data analysis?}
	``A Bayesian is one who, vaguely expecting a horse, and catching a glimpse of a donkey, strongly believes he has seen a mule.''
\end{frame}

\begin{frame}[fragile]{What is Bayesian data analysis?}
	\begin{columns}
		\begin{column}{0.7\linewidth}
			\begin{easylist}
					\ListProperties(Space=\listSpace, Space*=\listSpace)
					# Richard McElreath: ``Bayesian inference is just counting.''  
					# Count all the ways observed data could have arisen according to assumptions
					# Assumptions that can arise in more ways are more consistent with the data, and therefore more plausible
			\end{easylist}
		\end{column}
	\begin{column}{0.3\textwidth}
		\includegraphics[height=0.7\textheight]{figs/rethinking3.jpg}
	\end{column}
	\end{columns}
\end{frame}

\begin{frame}[fragile]{The Frequentist vs. Bayesian debacle}
		\begin{columns}
		\begin{column}{0.7\linewidth}
				\begin{easylist}
				\ListProperties(Space=\listSpace, Space*=\listSpace, Style2*=$\color{bggraydark}\blacktriangleright$\space)
				# Frequentist statistics
				## Probability defined as the limiting frequency at which events occur
				## Uncertainty arises from sampling variation
				# Bayesian statistics
				## Frequency and probability are different things
				## Uncertainty arises from our ignorance of the true state of the world
			\end{easylist}
		\end{column}
		\begin{column}{0.3\textwidth}
			\includegraphics[height=0.7\textheight]{figs/relevant_xkcd.png}
		\end{column}
	\end{columns}
\end{frame}

\begin{frame}[fragile]{A slide with a theorem and a proof.}
\begin{theorem}[Integral]
	\begin{equation*}
		\int_{a}^{b} f(x) \, dx = F(b) - F(a)
	\end{equation*}
\end{theorem}
\begin{proof}
Here's the proof.
\end{proof}
\end{frame}

% -------------------------------------------------------------------------
\section{Bayesian statistics}
\begin{frame}[fragile]{A slide with blocks}
	\begin{block}{title of the bloc}
	bloc text
	\end{block}
	
	\begin{exampleblock}{title of the bloc}
	bloc text
	\end{exampleblock}
\end{frame}

% -------------------------------------------------------------------------
\section{Markov Chains}
\begin{frame}[fragile]{A slide using pause}
	\begin{easylist}[itemize]
		\ListProperties(Space=\listSpace, Space*=\listSpace)
		# Represent Abelian groups on the computer \pause
		# Compute on Abelian groups \pause
		# Solve equations, factor group homomorphisms
	\end{easylist}
\end{frame}


\end{document}
